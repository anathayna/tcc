\chapter{Introdução}
\label{cap:01}

As redes sociais desempenham um papel significativo na vida cotidiana das pessoas, permitindo uma maior interatividade entre diferentes povos e grupos, encurtando a distância entre as pessoas e facilitando o intercâmbio cultural \cite{FortesTecBlog2021}. Embora apresentem muitos benefícios, elas também têm o poder de moldar as opiniões e crenças do público em todo o mundo. No documentário “O Dilema das Redes” \cite{SocialDilemma2020}, apresenta entrevistas com ex-funcionários de grandes empresas de tecnologia, como Facebook, Google e Twitter, que discutem temas sobre à coleta de dados pessoais, manipulação de usuários, disseminação de desinformação, teorias da conspiração e discurso de ódio.

% referenciar melhor a parte das redes sociais:
% https://academic.oup.com/jcmc/article/13/1/210/4583062 
% https://www.sciencedirect.com/science/article/abs/pii/S006524580901002X
% referenciar a parte "poder de moldar as opiniões e crenças do público em todo o mundo"

De acordo com Luiz Valério Trindade, ``o discurso de ódio se caracteriza pelas manifestações de pensamentos, valores e ideologias que buscam inferiorizar, desacreditar e humilhar uma pessoa ou um grupo social em função de suas características, como gênero, orientação sexual, filiação religiosa, raça, lugar de origem ou classe social'' \cite{Trindade2022}. Esse tipo de discurso pode se manifestar tanto verbalmente quanto por escrito e tem se tornado cada vez mais frequente nas plataformas de redes sociais, sendo expresso de forma explícita ou camuflada por meio de piadas ou memes.

% Um ataque direto ou indireto a pessoas com base em características, incluindo etnia, raça, nacionalidade, status de imigração, religião, casta, sexo, identidade de gênero, orientação sexual e deficiência ou doença. Definimos ataque como discurso violento ou desumano (comparando pessoas a coisas não humanas, por exemplo, animais), declarações de inferioridade e apelos à exclusão ou segregação. Zombar de crimes de ódio também é considerado discurso de ódio.

Um meme é uma unidade cultural que se espalha de pessoa para pessoa, geralmente na forma de imagens, vídeos, frases ou comportamentos que se tornam virais na internet. Os memes são frequentemente usados para transmitir ideias e valores com humor de uma maneira rápida e fácil de compartilhar\footnote{STEIN, T. O que é um meme (significado e exemplos engraçados). Disponível em: \url{https://www.dicionariopopular.com/meme/}. Acesso em: 9 de jun. de 2023.}. No entanto, alguns memes podem reforçar estereótipos e preconceitos \cite{Burke2004}, principalmente quando utilizam imagens ou frases que estereotipam grupos de pessoas com base em sua raça, gênero, sexualidade, religião e outros, contribuindo para a disseminação de discriminação e marginalização desses grupos.

% Richard Dawkins, em seu livro “O Gene Egoísta”, cunhou o termo “meme” para fazer uma analogia com o conceito de gene. Segundo ele, um meme é uma unidade de transmissão cultural ou de imitação. Em outras palavras, um meme é qualquer ideia, comportamento ou tendência que se propaga de pessoa para pessoa através da imitação ou da nossa herança cultural. Exemplos de memes incluem conceitos como religião, músicas populares e até mesmo modas. Atualmente, o termo “meme” também é usado para se referir a imagens, vídeos, bordões e hashtags que se tornam virais na internet. Dawkins, o “pai dos memes”, também se tornou um meme de internet, com suas frases satirizando o uso do termo e sua visão ateísta radical.

Embora muitos memes sejam inofensivos e divertidos, é fundamental lembrar que a maneira como as informações são transmitidas na internet pode ter um impacto real na maneira como as pessoas veem o mundo. Como é ilustrado no documentário, a disseminação do discurso de ódio nas mídias sociais pode levar à polarização social, à intolerância e, em casos extremos, até mesmo incitar crimes odiosos.

\section{Justificativa}

% \cite{spectrumieee2022}

Em 2020, a Meta divulgou avanços significativos em suas tecnologias de inteligência artificial (IA) para aprimorar a detecção de discurso de ódio em suas plataformas, como o Facebook e o Instagram. De acordo com o Relatório de Aplicação de Padrões da Comunidade divulgado, a IA é capaz de detectar 88,8\% do conteúdo de discurso de ódio que é removido de suas redes sociais \cite{MetaAIAdvances2020}. No mesmo ano, a empresa lançou um desafio aberto ao público\footnote{\textit{Hateful Memes Challenge and Dataset}. 2021. Disponível em: \url{https://ai.facebook.com/tools/hatefulmemes/}. Acesso em: 8 de mai. de 2023.}, com o objetivo de melhorar a automatização na identificação de discurso de ódio em memes, fomentando a pesquisa e os avanços de tecnologias de código aberto para a classificação multimodal, ou seja, memes com imagem e texto, pois, em alguns casos, um texto isolado da imagem pode não ser considerado ofensivo, mas quando combinados, pode ser classificado como discurso de ódio.

Em 2022, a Central Nacional de Denúncias \cite{Safernet2022} registrou um aumento significativo de 67,7\%, em comparação com o ano anterior, nas denúncias de crimes de discurso de ódio na internet, que evidência a importância da automação na identificação e contenção da disseminação dessas atividades criminosas.

De acordo com a \textit{Data Reportal}, já são 4,7 bilhões de usuários ativos nas redes sociais, o que representa 59\% da população mundial \cite{DataReportal2022}. Segundo Popolin, a reprodução de discursos de ódio está presente nas redes sociais, onde estereótipos e preconceitos são disfarçados de opiniões, com o escudo da liberdade de expressão. A facilidade e o comodismo das redes sociais da internet fazem com que os discursos de ódio sejam compartilhados e replicados em forma de memes \cite{POPOLIN2018}.

Os memes costumam ser sutis e dependem do contexto para determinar se seu conteúdo é ofensivo ou não. Embora possa ser fácil para os humanos identificar esses contextos em plataformas populares como Facebook e Twitter, é praticamente impossível realizar essa tarefa em uma grande quantidade de dados manualmente. 

O aprendizado de máquina e a inteligência artificial têm desempenhado um papel cada vez mais importante em diversos setores, sendo especialmente úteis para lidar com tarefas repetitivas e o processamento de grandes volumes de dados, quando aplicadas às redes sociais, podem contribuir significativamente para mitigar o compartilhamento em massa de informações que perpetuam problemas sociais emergentes. Com o desafio proposto pela Meta em 2020, foi vista a possibilidade de automatizar a detecção de discursos de ódio em memes, visando reduzir sua disseminação de forma mais eficiente, juntamente com a colaboração ativa da comunidade.


\section{Objetivo}

%\subsection{Objetivo Geral}

Desenvolver um sistema de classificação de memes que contenham discursos de ódio, utilizando o banco de dados de memes odiosos disponibilizado pela Meta. O sistema levará em conta as características multimodais dos memes, ou seja, a combinação de imagem e texto, e utilizará um modelo pré-treinado que combina técnicas de processamento de linguagem natural e visão computacional. O objetivo é alcançar alta precisão na identificação desses conteúdos ofensivos. Além disso, o projeto investigará padrões contextuais e culturais que possam influenciar a interpretação dos memes, contribuindo para um sistema mais robusto e sensível.



% Este documento e seu código-fonte são exemplos de referência de uso da classe
% \textsf{abntex2} e do pacote \textsf{abntex2cite}. O documento 
% exemplifica a elaboração de trabalho acadêmico (tese, dissertação e outros do
% gênero) produzido conforme a ABNT NBR 14724:2011 \emph{Informação e documentação
% - Trabalhos acadêmicos - Apresentação}.

% A expressão ``Modelo Canônico'' é utilizada para indicar que \abnTeX\ não é
% modelo específico de nenhuma universidade ou instituição, mas que implementa tão
% somente os requisitos das normas da ABNT. Uma lista completa das normas
% observadas pelo \abnTeX\ é apresentada em \citeonline{abntex2classe}.

% Sinta-se convidado a participar do projeto \abnTeX! Acesse o site do projeto em
% \url{http://abntex2.googlecode.com/}. Também fique livre para conhecer,
% estudar, alterar e redistribuir o trabalho do \abnTeX, desde que os arquivos
% modificados tenham seus nomes alterados e que os créditos sejam dados aos
% autores originais, nos termos da ``The \LaTeX\ Project Public
% License''\footnote{\url{http://www.latex-project.org/lppl.txt}}.

% Encorajamos que sejam realizadas customizações específicas deste exemplo para
% universidades e outras instituições --- como capas, folha de aprovação, etc.
% Porém, recomendamos que ao invés de se alterar diretamente os arquivos do
% \abnTeX, distribua-se arquivos com as respectivas customizações.
% Isso permite que futuras versões do \abnTeX~não se tornem automaticamente
% incompatíveis com as customizações promovidas. Consulte
% \citeonline{abntex2-wiki-como-customizar} par mais informações.

% Este documento deve ser utilizado como complemento dos manuais do \abnTeX\ 
% \cite{abntex2classe,abntex2cite,abntex2cite-alf} e da classe \textsf{memoir}
% \cite{memoir}. 

% Esperamos, sinceramente, que o \abnTeX\ aprimore a qualidade do trabalho que
% você produzirá, de modo que o principal esforço seja concentrado no principal:
% na contribuição científica.

% Equipe \abnTeX 

% Lauro César Araujo