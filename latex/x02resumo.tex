% resumo em português
\setlength{\absparsep}{18pt} % ajusta o espaçamento dos parágrafos do resumo
\begin{resumo}

O aprendizado de máquina e a inteligência artificial têm desempenhado um papel cada vez mais importante em diversos setores, essas tecnologias são especialmente úteis para lidar com tarefas repetitivas e com grandes volumes de dados, quando aplicadas às redes sociais, podem contribuir para mitigar o compartilhamento em massa de informações que perpetuam problemas sociais emergentes. Este projeto explora como essas ferramentas podem ser aplicadas para identificar discursos de ódio em memes, que geralmente são inofensivos e voltados para o humor, mas podem se tornar ofensivos dependendo do contexto e da combinação de imagens e texto. Com isso, pretende-se utilizar nesse trabalho o banco de dados de memes odiosos disponibilizado pela Meta, juntamente com o modelo CLIP (Contrastive Language-Image Pre-training), que é pré-treinado com conteúdo multimodal de imagem e texto. O objetivo principal é empregar o aprendizado de máquina para desenvolver um sistema capaz de distinguir e classificar memes que contenham discursos de ódio, visando alcançar uma alta acurácia.

 % Segundo a \citeonline[3.1-3.2]{NBR6028:2003}, o resumo deve ressaltar o objetivo, o método, os resultados e as conclusões do documento. A ordem e a extensão destes itens dependem do tipo de resumo (informativo ou indicativo) e do tratamento que cada item recebe no documento original. O resumo deve ser precedido da referência do documento, com exceção do resumo inserido no próprio documento. (\ldots) As palavras-chave devem figurar logo abaixo do resumo, antecedidas da expressão Palavras-chave:, separadas entre si por ponto e finalizadas também por ponto.

 \textbf{Palavras-chaves}: aprendizado de máquina. discurso de ódio. memes.
\end{resumo}

% outras tags:  Emotion Recognition,  Image Clustering,  Image Captioning, Meme Classification,  Misinformation,  Hateful Meme Classification

% resumo em inglês
% \begin{resumo}[Abstract]
%  \begin{otherlanguage*}{english}
%    Machine learning and artificial intelligence have played an increasingly important role in mitigating emerging social problems, and the beginning of this project showcases how it can be possible to identify hate speech in memes. Memes are generally harmless and focused on humor, but depending on the context, combining certain images and text can make a meme offensive. In this context, the dataset of hateful memes provided by Meta is utilized, along with the pre-trained multimodal VisualBERT model, and ensemble learning. The goal is to employ ensemble learning to classify memes as hateful, aiming to achieve an accuracy and AUROC (Area Under the Receiver Operating Characteristic Curve) exceeding 0.7.

%    \vspace{\onelineskip}
 
%    \noindent 
%    %\textbf{Key-words}: latex. abntex. text editoration.
%  \end{otherlanguage*}
% \end{resumo}