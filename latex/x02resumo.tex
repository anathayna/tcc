% resumo em português
\setlength{\absparsep}{18pt}
\begin{resumo}

O avanço do aprendizado de máquina e da inteligência artificial tem desempenhado um papel cada vez mais relevante em diversos setores. Essas tecnologias são especialmente úteis para lidar com tarefas repetitivas e com grandes volumes de dados, e quando aplicadas às redes sociais, podem auxiliar na mitigação do compartilhamento de conteúdos que reforçam problemas sociais emergente. Nesse contexto, este trabalho investiga o uso de técnicas de aprendizado multimodal para classificar discursos de ódio em memes, um formato de comunicação geralmente associado ao humor, mas que pode se tornar ofensivo a depender da combinação entre imagem e texto. Para isso, será utilizado o banco de dados de memes odiosos disponibilizado pela Meta, em conjunto com o modelo multimodal pré-treinado CLIP (\textit{Contrastive Language-Image Pre-training}), desenvolvido pela OpenAI. O objetivo principal é desenvolver um sistema capaz de classificar memes com discurso de ódio, aplicando a lógica \textit{fuzzy} como mecanismo complementar de interpretação dos resultados, permitindo uma classificação gradual do grau de discurso de ódio. A eficácia do modelo será avaliada com base em métricas como acurácia e taxa de detecção, permitindo a comparação com trabalhos de referência e demonstrando o potencial da abordagem proposta para a moderação automatizada de memes.

\textbf{Palavras-chaves}: discurso de ódio. memes. apredizado de máquina. aprendizado multimodal. lógica fuzzy.
\end{resumo}
