% resumo em português
\setlength{\absparsep}{18pt} % ajusta o espaçamento dos parágrafos do resumo
\begin{resumo}

O avanço do aprendizado de máquina e a inteligência artificial tem desempenhado um papel cada vez mais relevante em diversos setores. Essas tecnologias são especialmente úteis para lidar com tarefas repetitivas e com grandes volumes de dados, e quando aplicadas às redes sociais, podem contribuir para mitigar o compartilhamento em massa de conteúdos que reforçam problemas sociais emergentes. Nesse contexto, este trabalho explora o uso de técnicas de aprendizado multimodal para identificar discursos de ódio em memes, um formato de comunicação geralmente associado ao humor, mas que pode se tornar ofensivo a depender da combinação entre imagem e texto. Para isso, será utilizado o banco de dados de memes odiosos disponibilizado pela Meta, em conjunto com o modelo CLIP (\textit{Contrastive Language-Image Pre-training}). O objetivo principal é desenvolver um sistema capaz de classificar memes com discurso de ódio, aplicando a lógica \textit{fuzzy} como mecanismo complementar de interpretação dos resultados, permitindo uma classificação gradual do grau de discurso de ódio. A eficácia do modelo será medida por métricas como acurácia e taxa de detecção, possibilitando a comparação com trabalhos de referência e destacando o potencial da integração entre aprendizado multimodal e lógica \textit{fuzzy} na moderação de conteúdo nas redes sociais.


 % Segundo a \citeonline[3.1-3.2]{NBR6028:2003}, o resumo deve ressaltar o objetivo, o método, os resultados e as conclusões do documento. A ordem e a extensão destes itens dependem do tipo de resumo (informativo ou indicativo) e do tratamento que cada item recebe no documento original. O resumo deve ser precedido da referência do documento, com exceção do resumo inserido no próprio documento. (\ldots) As palavras-chave devem figurar logo abaixo do resumo, antecedidas da expressão Palavras-chave:, separadas entre si por ponto e finalizadas também por ponto.

 \textbf{Palavras-chaves}: discurso de ódio. memes. apredizado de máquina. multimodal. lógica fuzzy.
\end{resumo}

% outras tags:  Emotion Recognition,  Image Clustering,  Image Captioning, Meme Classification,  Misinformation,  Hateful Meme Classification

% resumo em inglês
% \begin{resumo}[Abstract]
%  \begin{otherlanguage*}{english}
%    Machine learning and artificial intelligence have played an increasingly important role in mitigating emerging social problems, and the beginning of this project showcases how it can be possible to identify hate speech in memes. Memes are generally harmless and focused on humor, but depending on the context, combining certain images and text can make a meme offensive. In this context, the dataset of hateful memes provided by Meta is utilized, along with the pre-trained multimodal VisualBERT model, and ensemble learning. The goal is to employ ensemble learning to classify memes as hateful, aiming to achieve an accuracy and AUROC (Area Under the Receiver Operating Characteristic Curve) exceeding 0.7.

%    \vspace{\onelineskip}
 
%    \noindent 
%    %\textbf{Key-words}: latex. abntex. text editoration.
%  \end{otherlanguage*}
% \end{resumo}