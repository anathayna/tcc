%% abtex2-modelo-trabalho-academico.tex, v-1.9.2 laurocesar
%% Copyright 2012-2014 by abnTeX2 group at http://abntex2.googlecode.com/ 
%%
%% This work may be distributed and/or modified under the
%% conditions of the LaTeX Project Public License, either version 1.3
%% of this license or (at your option) any later version.
%% The latest version of this license is in
%%   http://www.latex-project.org/lppl.txt
%% and version 1.3 or later is part of all distributions of LaTeX
%% version 2005/12/01 or later.
%%
%% This work has the LPPL maintenance status `maintained'.
%% 
%% The Current Maintainer of this work is the abnTeX2 team, led
%% by Lauro César Araujo. Further information are available on 
%% http://abntex2.googlecode.com/
%%
%% This work consists of the files abntex2-modelo-trabalho-academico.tex,
%% abntex2-modelo-include-comandos and abntex2-modelo-references.bib
%%

% ------------------------------------------------------------------------
% ------------------------------------------------------------------------
% abnTeX2: Modelo de Trabalho Academico (tese de doutorado, dissertacao de
% mestrado e trabalhos monograficos em geral) em conformidade com 
% ABNT NBR 14724:2011: Informacao e documentacao - Trabalhos academicos -
% Apresentacao
% ------------------------------------------------------------------------
% ------------------------------------------------------------------------

\documentclass[
	% -- opções da classe memoir --
	12pt,				% tamanho da fonte
	openright,			% capítulos começam em pág ímpar (insere página vazia caso preciso)
	twoside,			% para impressão em verso e anverso. Oposto a oneside
	a4paper,			% tamanho do papel. 
	% -- opções da classe abntex2 --
	%chapter=TITLE,		% títulos de capítulos convertidos em letras maiúsculas
	%section=TITLE,		% títulos de seções convertidos em letras maiúsculas
	%subsection=TITLE,	% títulos de subseções convertidos em letras maiúsculas
	%subsubsection=TITLE,% títulos de subsubseções convertidos em letras maiúsculas
	% -- opções do pacote babel --
	english,			% idioma adicional para hifenização
	french,				% idioma adicional para hifenização
	spanish,			% idioma adicional para hifenização
	brazil				% o último idioma é o principal do documento
	]{abntex2}

% ---
% Pacotes básicos 
% ---
\usepackage{lmodern}			% Usa a fonte Latin Modern			
\usepackage[T1]{fontenc}		% Selecao de codigos de fonte.
\usepackage[utf8]{inputenc}		% Codificacao do documento (conversão automática dos acentos)
\usepackage{lastpage}			% Usado pela Ficha catalográfica
\usepackage{indentfirst}		% Indenta o primeiro parágrafo de cada seção.
\usepackage{color}				% Controle das cores
\usepackage{graphicx}			% Inclusão de gráficos
\usepackage{microtype} 			% para melhorias de justificação
% ---
		
% ---
% Pacotes adicionais, usados apenas no âmbito do Modelo Canônico do abnteX2
% ---
\usepackage{lipsum}				% para geração de dummy text
% ---

% ---
% Pacotes de citações
% ---
\usepackage[brazilian,hyperpageref]{backref}	 % Paginas com as citações na bibl
\usepackage[alf,abnt-emphasize=bf]{abntex2cite}	% Citações padrão ABNT

% --- 
% CONFIGURAÇÕES DE PACOTES
% --- 

% ---
% Configurações do pacote backref
% Usado sem a opção hyperpageref de backref
\renewcommand{\backrefpagesname}{Citado na(s) página(s):~}
% Texto padrão antes do número das páginas
\renewcommand{\backref}{}
% Define os textos da citação
\renewcommand*{\backrefalt}[4]{
	\ifcase #1 %
		Nenhuma citação no texto.%
	\or
		Citado na página #2.%
	\else
		Citado #1 vezes nas páginas #2.%
	\fi}%
% ---

% ---
% Informações de dados para CAPA e FOLHA DE ROSTO
% ---
\titulo{Classificação de discurso de ódio em memes com aprendizado multimodal}
\autor{Ana Thayna Conceição França}
\local{SP - Brasil}
\data{2025}
\orientador{...}
\coorientador{...}
\instituicao{%
  Centro Universitário Senac - Santo Amaro
  \par
  Bacharelado em Ciência da Computação}
\tipotrabalho{Monografia}
% O preambulo deve conter o tipo do trabalho, o objetivo, 
% o nome da instituição e a área de concentração 
\preambulo{Monografia apresentada na disciplina Trabalho de Conclusão de Curso, como parte dos
requisitos para obtenção do título de Bacharel em Ciência da Computação.}
% ---



% ---
% Configurações de aparência do PDF final

% alterando o aspecto da cor azul
\definecolor{blue}{RGB}{41,5,195}

% informações do PDF
\makeatletter
\hypersetup{
     	%pagebackref=true,
		pdftitle={\@title}, 
		pdfauthor={\@author},
    	pdfsubject={\imprimirpreambulo},
	    pdfcreator={LaTeX with abnTeX2},
		pdfkeywords={abnt}{latex}{abntex}{abntex2}{trabalho acadêmico}, 
		colorlinks=true,       		% false: boxed links; true: colored links
    	linkcolor=blue,          	% color of internal links
    	citecolor=blue,        		% color of links to bibliography
    	filecolor=magenta,      		% color of file links
		urlcolor=blue,
		bookmarksdepth=4
}
\makeatother
% --- 

% --- 
% Espaçamentos entre linhas e parágrafos 
% --- 

% O tamanho do parágrafo é dado por:
\setlength{\parindent}{1.3cm}

% Controle do espaçamento entre um parágrafo e outro:
\setlength{\parskip}{0.2cm}  % tente também \onelineskip

% ---
% compila o indice
% ---
\makeindex
% ---

% ----
% Início do documento
% ----
\begin{document}

% Retira espaço extra obsoleto entre as frases.
\frenchspacing 

% ----------------------------------------------------------
% ELEMENTOS PRÉ-TEXTUAIS
% ----------------------------------------------------------
% \pretextual

% ---
% Capa
% ---
\imprimircapa
% ---

% ---
% Folha de rosto
% (o * indica que haverá a ficha bibliográfica)
% ---
%\imprimirfolhaderosto*
% ---

% ---
% Dedicatória
% ---
%\input{x00dedicatoria}
% ---

% ---
% Agradecimentos
% ---
%\input{x01agradecimentos}
% ---

% ---
% RESUMOS
% ---
% resumo em português
\setlength{\absparsep}{18pt}
\begin{resumo}

O avanço do aprendizado de máquina e da inteligência artificial tem desempenhado um papel cada vez mais relevante em diversos setores. Essas tecnologias são especialmente úteis para lidar com tarefas repetitivas e com grandes volumes de dados, e quando aplicadas às redes sociais, podem auxiliar na mitigação do compartilhamento de conteúdos que reforçam problemas sociais emergente. Nesse contexto, este trabalho investiga o uso de técnicas de aprendizado multimodal para classificar discursos de ódio em memes, um formato de comunicação geralmente associado ao humor, mas que pode se tornar ofensivo a depender da combinação entre imagem e texto. Para isso, será utilizado o banco de dados de memes odiosos disponibilizado pela Meta, em conjunto com o modelo multimodal pré-treinado CLIP (\textit{Contrastive Language-Image Pre-training}), desenvolvido pela OpenAI. O objetivo principal é desenvolver um sistema capaz de classificar memes com discurso de ódio, aplicando a lógica \textit{fuzzy} como mecanismo complementar de interpretação dos resultados, permitindo uma classificação gradual do grau de discurso de ódio. A eficácia do modelo será avaliada com base em métricas como acurácia e taxa de detecção, permitindo a comparação com trabalhos de referência e demonstrando o potencial da abordagem proposta para a moderação automatizada de memes.

\textbf{Palavras-chaves}: discurso de ódio. memes. apredizado de máquina. aprendizado multimodal. lógica fuzzy.
\end{resumo}

% ---

% ---
% inserir lista de ilustrações
% ---
\input{x03ilustracoes}
% ---

% ---
% inserir lista de tabelas
% ---
%\input{x04tabelas}
% ---

% ---
% inserir lista de abreviaturas e siglas
% ---
\begin{siglas}
  \item[AUROC] Área sob a Curva Característica de Operação do Receptor
  \item[BERT] \textit{Bidirecional Encoder Representations from Transformer}
  \item[IA] Inteligência Artificial
  \item[MMF] \textit{MultiModal Framework}
  \item[PLN] Processamento de Linguagem Natural
\end{siglas}
% ---

% ---
% inserir lista de símbolos
% ---
%\input{x06simbolos}
% ---

% ---
% inserir o sumario
% ---
\pdfbookmark[0]{\contentsname}{toc}
\tableofcontents*
\cleardoublepage
% ---



% ----------------------------------------------------------
% ELEMENTOS TEXTUAIS
% ----------------------------------------------------------
\textual
\chapter{Introdução}
\label{cap:01}

As redes sociais digitais tem desempenhado um papel significativo na vida cotidiana das pessoas, permitindo uma maior interatividade entre diferentes povos e grupos, encurtando a distância entre as pessoas e facilitando o intercâmbio cultural \cite{Nascimento2017}. Embora apresentem muitos benefícios, elas também têm o poder de moldar as opiniões e crenças do público em todo o mundo. No documentário “O Dilema das Redes” \cite{SocialDilemma2020}, apresenta entrevistas com ex-funcionários de grandes empresas de tecnologia, como Facebook, Google e Twitter, que discutem temas sobre à coleta de dados pessoais, manipulação de usuários, disseminação de desinformação, teorias da conspiração e discurso de ódio.

% referenciar melhor a parte das redes sociais:
% https://www.amazon.com.br/Social-Networking-Redefining-Communication-Digital/dp/1611477409
% https://www.sciencedirect.com/science/article/abs/pii/S006524580901002X
% referenciar a parte "poder de moldar as opiniões e crenças do público em todo o mundo"

De acordo com Luiz Valério Trindade, ``o discurso de ódio se caracteriza pelas manifestações de pensamentos, valores e ideologias que buscam inferiorizar, desacreditar e humilhar uma pessoa ou um grupo social em função de suas características, como gênero, orientação sexual, filiação religiosa, raça, lugar de origem ou classe social'' \cite{Trindade2022}. Esse tipo de discurso pode se manifestar tanto verbalmente quanto por escrito, e tem se tornado cada vez mais frequente nas plataformas de redes sociais, sendo expresso de forma explícita ou camuflada por meio de piadas ou memes.

% Um ataque direto ou indireto a pessoas com base em características, incluindo etnia, raça, nacionalidade, status de imigração, religião, casta, sexo, identidade de gênero, orientação sexual e deficiência ou doença. Definimos ataque como discurso violento ou desumano (comparando pessoas a coisas não humanas, por exemplo, animais), declarações de inferioridade e apelos à exclusão ou segregação. Zombar de crimes de ódio também é considerado discurso de ódio.

Um meme é uma unidade cultural que se propaga de pessoa para pessoa, geralmente na forma de imagens, vídeos, frases ou comportamentos que se tornam virais na internet. Segundo o biólogo Richard Dawkins, em seu livro O Gene Egoísta \cite{Dawkins1976}, um meme é ''uma unidade de transmissão cultural, ou de imitação''. Os memes são frequentemente utilizados para transmitir ideias e valores com humor, de maneira rápida e fácil de compartilhar. Entretanto, alguns memes podem reforçar estereótipos e preconceitos, principalmente quando utilizam imagens ou frases que estereotipam grupos de pessoas com base em sua raça, gênero, sexualidade, religião e outros, contribuindo para a disseminação de discriminação e marginalização desses grupos \cite{Burke2004}.

% Richard Dawkins, em seu livro “O Gene Egoísta”, cunhou o termo “meme” para fazer uma analogia com o conceito de gene. Segundo ele, um meme é uma unidade de transmissão cultural ou de imitação. Em outras palavras, um meme é qualquer ideia, comportamento ou tendência que se propaga de pessoa para pessoa através da imitação ou da nossa herança cultural. Exemplos de memes incluem conceitos como religião, músicas populares e até mesmo modas. Atualmente, o termo “meme” também é usado para se referir a imagens, vídeos, bordões e hashtags que se tornam virais na internet. Dawkins, o “pai dos memes”, também se tornou um meme de internet, com suas frases satirizando o uso do termo e sua visão ateísta radical.

Embora muitos memes sejam inofensivos e divertidos, é fundamental lembrar que a maneira como as informações são transmitidas na internet pode ter um impacto real na maneira como as pessoas veem o mundo. Como é ilustrado no documentário, a disseminação do discurso de ódio nas mídias sociais pode levar à polarização social, à intolerância e, em casos extremos, até mesmo incitar crimes odiosos.

\section{Justificativa}

% \cite{spectrumieee2022}

Em 2020, a Meta divulgou avanços significativos em suas tecnologias de inteligência artificial (IA) para aprimorar a detecção de discurso de ódio em suas plataformas, como o Facebook e o Instagram. De acordo com o Relatório de Aplicação de Padrões da Comunidade divulgado, a IA é capaz de detectar 88,8\% do conteúdo de discurso de ódio que é removido de suas redes sociais \cite{MetaAIAdvances2020}. No mesmo ano, a empresa lançou um desafio aberto ao público\footnote{\textit{Hateful Memes Challenge and Dataset}. 2021. Disponível em: \url{https://ai.facebook.com/tools/hatefulmemes/}. Acesso em: 8 de mai. de 2023.}, com o objetivo de melhorar a automatização na identificação de discurso de ódio em memes, fomentando a pesquisa e os avanços de tecnologias de código aberto para a classificação multimodal, ou seja, memes com imagem e texto, pois, em alguns casos, um texto isolado da imagem pode não ser considerado ofensivo, mas quando combinados, pode ser classificado como discurso de ódio.

Em 2022, a Central Nacional de Denúncias \cite{Safernet2022} registrou um aumento significativo de 67,7\%, em comparação com o ano anterior, nas denúncias de crimes de discurso de ódio na internet, que evidência a importância da automação na identificação e contenção da disseminação dessas atividades criminosas.

De acordo com a \textit{Data Reportal}, já são 4,7 bilhões de usuários ativos nas redes sociais, o que representa 59\% da população mundial \cite{DataReportal2022}. Segundo Popolin, a reprodução de discursos de ódio está presente nas redes sociais, onde estereótipos e preconceitos são disfarçados de opiniões, com o escudo da liberdade de expressão. A facilidade e o comodismo das redes sociais da internet fazem com que os discursos de ódio sejam compartilhados e replicados em forma de memes \cite{POPOLIN2018}.

Os memes costumam ser sutis e dependem do contexto para determinar se seu conteúdo é ofensivo ou não. Embora possa ser fácil para os humanos identificar esses contextos em plataformas populares como Facebook e Twitter, é praticamente impossível realizar essa tarefa em uma grande quantidade de dados manualmente. 

O aprendizado de máquina e a inteligência artificial têm desempenhado um papel cada vez mais importante em diversos setores, sendo especialmente úteis para lidar com tarefas repetitivas e o processamento de grandes volumes de dados, quando aplicadas às redes sociais, podem contribuir significativamente para mitigar o compartilhamento em massa de informações que perpetuam problemas sociais emergentes. Com o desafio proposto pela Meta em 2020, foi vista a possibilidade de automatizar a detecção de discursos de ódio em memes, visando reduzir sua disseminação de forma mais eficiente, juntamente com a colaboração ativa da comunidade.


\section{Objetivo}

%\subsection{Objetivo Geral}

Desenvolver um sistema de classificação de memes que contenham discursos de ódio, utilizando o banco de dados de memes odiosos disponibilizado pela Meta. O sistema levará em conta as características multimodais dos memes, ou seja, a combinação de imagem e texto, e utilizará um modelo pré-treinado que combina técnicas de processamento de linguagem natural e visão computacional. O objetivo é alcançar alta precisão na identificação desses conteúdos ofensivos. Além disso, o projeto investigará padrões contextuais e culturais que possam influenciar a interpretação dos memes, contribuindo para um sistema mais robusto e sensível.



% Este documento e seu código-fonte são exemplos de referência de uso da classe
% \textsf{abntex2} e do pacote \textsf{abntex2cite}. O documento 
% exemplifica a elaboração de trabalho acadêmico (tese, dissertação e outros do
% gênero) produzido conforme a ABNT NBR 14724:2011 \emph{Informação e documentação
% - Trabalhos acadêmicos - Apresentação}.

% A expressão ``Modelo Canônico'' é utilizada para indicar que \abnTeX\ não é
% modelo específico de nenhuma universidade ou instituição, mas que implementa tão
% somente os requisitos das normas da ABNT. Uma lista completa das normas
% observadas pelo \abnTeX\ é apresentada em \citeonline{abntex2classe}.

% Sinta-se convidado a participar do projeto \abnTeX! Acesse o site do projeto em
% \url{http://abntex2.googlecode.com/}. Também fique livre para conhecer,
% estudar, alterar e redistribuir o trabalho do \abnTeX, desde que os arquivos
% modificados tenham seus nomes alterados e que os créditos sejam dados aos
% autores originais, nos termos da ``The \LaTeX\ Project Public
% License''\footnote{\url{http://www.latex-project.org/lppl.txt}}.

% Encorajamos que sejam realizadas customizações específicas deste exemplo para
% universidades e outras instituições --- como capas, folha de aprovação, etc.
% Porém, recomendamos que ao invés de se alterar diretamente os arquivos do
% \abnTeX, distribua-se arquivos com as respectivas customizações.
% Isso permite que futuras versões do \abnTeX~não se tornem automaticamente
% incompatíveis com as customizações promovidas. Consulte
% \citeonline{abntex2-wiki-como-customizar} par mais informações.

% Este documento deve ser utilizado como complemento dos manuais do \abnTeX\ 
% \cite{abntex2classe,abntex2cite,abntex2cite-alf} e da classe \textsf{memoir}
% \cite{memoir}. 

% Esperamos, sinceramente, que o \abnTeX\ aprimore a qualidade do trabalho que
% você produzirá, de modo que o principal esforço seja concentrado no principal:
% na contribuição científica.

% Equipe \abnTeX 

% Lauro César Araujo
% open pretrained transformer (OPT)
% Gabriel Goh et al. Multimodal Neurons in Artificial Neural Networks. https://openai.com/blog/multimodal-neurons/. 2021.

% https://paperswithcode.com/paper/exploring-contrastive-learning-for-multimodal

% thesis https://rizavelioglu.github.io/files/RizaVelioglu-MScThesis.pdf

% visualBERT é gratuito? não tenho certeza
% ajuste fino é feito a cada atualização? que atualização é essa?
% taxa de aprendizado explicar

\chapter{Revisão Bibliográfica}
\label{cap:02}

Neste capítulo, será feita uma revisão da fundamentação teórica necessária utilizada ao longo deste projeto, visando alcançar o objetivo proposto. % apresentação genérica

\section{Inteligência Artificial}

A inteligência artificial é uma área de estudo da ciência da computação que busca compreender e simular a inteligência humana a fim de reproduzi-la artificialmente em uma máquina \cite{Haugeland1985}. Ela envolve o uso de algoritmos, sistemas e técnicas que permitem que as máquinas aprendam e realizem tarefas que normalmente exigiriam a inteligência humana.

Para desenvolver a IA, usamos técnicas que se associam as funções cognitivas da mente humana. Como o uso de redes neurais que buscam replicar o funcionamento dos neurônios no cérebro humano. O aprendizado de máquina, que treina computadores para desempenhar tarefas semelhantes às realizadas por seres humanos. Isso inclui as habilidades como planejamento, compreensão, aprendizado, raciocínio, resolução de problemas e tomada de decisões com base em dados e informações fornecidas.

A IA é um campo multidisciplinar que combina conhecimentos da ciência da computação, da matemática, da estatística, da neurociência e de outras áreas para criar programas e sistemas capazes de aprender, raciocinar, reconhecer padrões, tomar decisões e resolver problemas de forma autônoma. Existem diferentes abordagens, técnicas e aplicações da inteligência artificial, mas o aprendizado de máquina é uma das bases fundamentais para seu funcionamento e, consequentemente, para realização deste trabalho.

\section{Aprendizado de Máquina}

O aprendizado de máquina é uma aplicação na qual os sistemas são treinados com grandes conjuntos de dados para reconhecer padrões, fazer previsões e tomar decisões com base nesses padrões \cite{HoschMachineLearning}. Trata-se de um método de análise de dados que utiliza uma abordagem estatística e computacional, permitindo que os sistemas aprendam com os dados disponíveis e com o mínimo de intervenção humana necessária, um exemplo é a moderação do conteúdo postado em redes sociais, com a finalidade de classificar se um determinado conteúdo é ofensivo ou não.

% Nesse cenário, o treinamento e o pré-treinamento são etapas cruciais no processo de desenvolvimento de modelos capazes de aprender a partir dos dados.

% O treinamento refere-se à fase em que um modelo de aprendizado de máquina é alimentado com um conjunto de dados, onde é exposto a exemplos e realiza iterações para ajustar seus parâmetros internos, de modo que seja capaz de fazer previsões ou tomar decisões com base nos dados de entrada, aprendendo padrões e relações, a fim de realizar tarefas específicas.

% Por outro lado, o pré-treinamento adota uma abordagem em que o modelo é inicialmente treinado em um grande conjunto de dados não rotulados, onde não há informações explícitas ou respostas corretas. Isso permiti que o modelo aprenda a capturar características gerais e relevantes dos dados, estabelecendo uma base inicial de conhecimento antes de passar pelo treinamento convencional com os dados rotulados. Em seguida, o modelo é refinado através do processo de ajuste fino (do inglês, \textit{fine-tuning}), utilizando um conjunto menor de dados rotulados para adaptar-se a uma tarefa específica.

Existem várias abordagens de aprendizado de máquina, sendo o aprendizado supervisionado, as redes neurais, o \textit{deep learning} e o aprendizado por conjuntos de grande importância para a execução deste trabalho.

\subsection{Aprendizado supervisionado}

O aprendizado supervisionado utiliza um conjuntos de dados rotulados para treinar algoritmos com o objetivo de aprender a classificar os dados ou prever resultados com precisão com base nas informações fornecidas nos rótulos dos dados de treinamento. O modelo realiza ajustes finos (do inglês, \textit{fine-tuning}) à medida que os dados de entrada são inseridos, visando aprender a mapear corretamente os dados para as saídas desejadas. Isso permite a previsão ou classificação de novos exemplos não rotulados com base no aprendizado adquirido \cite{RussellNorvig1995}.

% Durante o ajuste fino, os parâmetros do modelo são atualizados para se adaptarem aos dados rotulados e melhorarem o desempenho em relação à tarefa específica.

%O processo de fine-tuning permite que o modelo se beneficie do conhecimento prévio adquirido durante o pré-treinamento, enquanto se adapta aos dados específicos da tarefa atual. Isso pode resultar em um modelo mais eficiente e preciso na detecção de discurso de ódio em memes, pois combina a capacidade de generalização do modelo pré-treinado com a especialização nos dados rotulados relevantes para a tarefa específica.

% \footnote{\textit{What is machine learning?}. 2016. Disponível em: \url{https://www.ibm.com/topics/machine-learning}. Acesso em: 30 de mai. de 2023.}.

% O papel dos algoritmos de aprendizado supervisionado é procurar em um espaço de hipóteses [10] uma hipótese adequada que permita boas previsões para um problema específico. Mesmo que o espaço de hipóteses inclua hipóteses adequadas a um problema específico, pode ser difícil encontrar uma que seja boa. Para formar uma hipótese mais forte, os ensembles combinam várias hipóteses. O termo “conjunto” refere-se a abordagens que usam o mesmo aprendiz básico para produzir várias hipóteses. Em outras palavras, o aprendizado conjunto visa aumentar a capacidade de generalização e a robustez em um único modelo, combinando as previsões de vários modelos básicos. Especificamente, usamos a técnica de votação majoritária (também conhecida como votação forte ou classificador de votação4), que combina vários classificadores e prevê rótulos de classe usando um voto majoritário. O classificador resultante costuma ser usado para equilibrar as deficiências de vários modelos com desempenho igualmente bom. Como resultado, ele tem um desempenho melhor do que qualquer outro modelo usado no conjunto.

\subsection{Redes Neurais}

% redes neurais profundas
% redes neurais recorrentes ou convolucionais complexas

As redes neurais são uma analogia neurobiológica inspirada no funcionamento do sistema nervoso do cérebro humano, capazes de adquirir conhecimento através de um processo de aprendizagem a partir de dados de entrada, com ou sem supervisão. Essas redes são capazes de armazenar o conhecimento adquirido interligando células computacionais simples denominadas ``neurônios'' ou ``unidades de processamento''. Com essa estrutura interconectada, e com o treinamento apropriado, são capazes de modelar, analisar e reconhecer padrões, resultando na produção de conhecimento de maneira distribuída \cite{Haykin2007}.

\begin{figure*}[!htbp]
	\centering
	\includegraphics[scale=0.4]{imagens/arch-rede-neural-artificial.png}
    \caption {Arquitetura de uma Rede Neural Artificial \cite{RedeNeuralImagem}.}
\end{figure*}

% autor da imagem: https://www.researchgate.net/figure/Artificial-neural-network-architecture-ANN-i-h-1-h-2-h-n-o_fig1_321259051

Elas são compostas por um conjunto interconectado de neurônios artificiais, ou ``nós'', que são organizados em várias camadas, sendo a de entrada responsável por receber os dados, as intermediárias (também conhecidas como camadas ocultas) responsáveis por processar as informações e a de saída responsável por produzir as saídas finais com os resultados da rede neural.

Cada neurônio artificial possui pesos e um valor de viés, sendo o primeiro representando a importância da conexão entre os neurônios, enquanto o segundo é uma medida flexível que facilita a ativação do neurônio de acordo com as preferências incorporadas. Durante o treinamento, os pesos e vieses são ajustados para otimizar o desempenho da rede. A rede funciona por meio de dois processos principais, a propagação direta (do inglês, \textit{forward propagation}), e a retropropagação do erro (do inglês, \textit{backpropagation}). Na propagação direta, os dados de entrada são alimentados na rede neural, passando pelos neurônios e pelas camadas, e resultando em uma saída. Cada neurônio combina linearmente os valores de entrada ponderados pelos pesos, aplica uma função de ativação e repassa o resultado para os neurônios da próxima camada. O erro é calculado em relação aos valores desejados, e a retropropagação do erro ajusta os pesos e vieses dos neurônios para minimizar o erro. Esse processo é repetido até que os pesos sejam ajustados para produzir resultados desejáveis.

Elas são amplamente utilizadas em áreas como visão computacional e processamento de linguagem natural devido à sua capacidade de aprender a partir dos dados e fornecer soluções complexas para problemas desafiadores, capazes de extrair características relevantes dos dados e realizar tarefas como classificação e reconhecimento de padrões.

% Redes neurais recorrentes (RNNs), nas quais os dados podem fluir em qualquer direção, são usadas para aplicações como modelagem de linguagem.[141][142][143][144][145]

% Redes neurais profundas convolucionais (CNNs) são usadas em visão computacional.[147]

% No contexto da detecção de discurso de ódio em memes, pode-se utilizar um modelo pré-treinado, como uma rede neural convolucional (CNN) ou uma rede neural de transformador (por exemplo, BERT), que foi treinado em um grande conjunto de dados em uma tarefa relacionada, como classificação de imagens ou processamento de linguagem natural. Em seguida, o modelo pré-treinado é ajustado com dados rotulados específicos para a detecção de discurso de ódio em memes, refinando seus parâmetros por meio do aprendizado supervisionado.

\subsection{Aprendizado profundo}

O aprendizado profundo (do inglês, \textit{deep learning}) é uma abordagem que utiliza redes neurais artificiais profundas que utilizam múltiplas camadas de processamento interconectadas para extrair representações complexas dos dados com vários níveis de abstração e reconhecimento de padrões. Essas redes neurais profundas são capazes de aprender de forma hierárquica, com cada camada representando características mais abstratas à medida que se aprofunda na rede \cite{AurelienGeron2019}. Essa abordagem permite que o aprendizado profundo seja capaz de lidar com problemas de maior complexidade e extraia automaticamente características relevantes dos dados.

O aprendizado profundo e as redes neurais tem impulsionado significativamente o avanço em áreas como a visão computacional e o processamento de linguagem natural.

\subsection{Aprendizado por conjuntos}

% qual é a diferencia entre ensemble e clustering?
% o ensemble learning envolve a combinação de modelos para melhorar a precisão das previsões, enquanto o clustering visa agrupar dados com base em suas similaridades. Enquanto o ensemble learning lida com tarefas de previsão, o clustering se concentra na descoberta de estruturas e padrões nos dados.

O aprendizado por conjuntos (do inglês, \textit{ensemble learning}) é uma técnica que combina múltiplos modelos de aprendizado de máquina para melhorar o desempenho e a precisão das previsões \cite{OPITZEnsemble}. Ao invés de depender de um único modelo, o aprendizado por conjuntos aproveita a diversidade e a opinião coletiva dos modelos individuais para tomar decisões mais precisas, pois os erros cometidos por um modelo podem ser compensados por outros mais precisos. 

Uma abordagem comum no \textit{ensemble learning} é o \textbf{voto majoritário}, em que cada modelo do conjunto faz uma previsão para uma determinada instância de dados, conhecida como classe, e a classe mais frequente entre os modelos é escolhida como a previsão final. O voto majoritário é simples e eficaz, especialmente quando os modelos individuais têm uma taxa de acerto significativa. Outras estratégias também podem ser utilizadas, como a combinação de probabilidades ou o uso de pesos para ponderar as previsões dos modelos individuais. Essas abordagens levam em consideração a confiança ou a qualidade dos modelos, o que contribui para um aumento geral na precisão das previsões.

Outra abordagem importante de destacar é o uso dos \textbf{hiperparâmetros}, que são parâmetros configuráveis que afetam o desempenho e o comportamento do modelo, influenciando fatores como a complexidade, a velocidade de treinamento e a capacidade de generalização. No \textit{ensemble learning}, os hiperparâmetros são usados para selecionar os modelos de melhor desempenho para o conjunto por meio da busca em grade, que envolve uma grade de valores de hiperparâmetros onde é feita a pesquisa de todas as combinações possíveis desses valores. A seleção adequada dos hiperparâmetros é fundamental para otimizar o desempenho do \textit{ensemble learning} e obter os melhores resultados, da qual requer ajustes dos valores dos hiperparâmetros e avaliação do desempenho do modelo usando métricas apropriadas. Tais métricas são essenciais para avaliar e mensurar os resultados obtidos, sendo elas a acurácia e a AUROC.

% levando em conta o número de modelos no conjunto, o tipo de modelo utilizado, os critérios de combinação dos modelos, os pesos atribuídos a cada modelo, entre outros.

% uma pesquisa de grade é realizada em uma faixa de valores para a taxa de aprendizado, tamanho do lote, aquecimento e número de iterações. Os resultados do ajuste de hiperparâmetros são então calculados sobre três sementes aleatórias, juntamente com seu desvio padrão, para obter uma estimativa mais robusta do desempenho do modelo.

% o processo de aprendizado do conjunto, uma pesquisa de hiperparâmetro é realizada para encontrar a melhor configuração para os parâmetros, incluindo taxa de aprendizado, etapas de aquecimento, tipo de aquecimento, fator de aquecimento e iterações de aquecimento. 

% Exemplos comuns de hiperparâmetros incluem a taxa de aprendizado, o número de camadas ocultas em uma rede neural, o tamanho do lote de treinamento.

A \textbf{acurácia} é uma métrica que avalia a performance de um modelo de aprendizado de máquina, representando a proporção de previsões corretas em relação ao total de previsões realizadas pelo modelo. É calculada dividindo o número de previsões corretas pelo número total de exemplos no conjunto de dados. No entanto, em casos de desbalanceamento de dados, quando uma classe é mais frequente que as outras, a acurácia por si só não é a métrica mais adequada. Nesses casos, é importante considerar outras métricas que possam fornecer uma visão mais abrangente do desempenho do modelo, como a AUROC.

% colocar uma imagem aqui

A \textbf{AUROC} (Área sob a Curva Característica de Operação do Receptor) é uma métrica utilizada principalmente para avaliar o desempenho de modelos de classificação binária, como os utilizados em problemas de detecção de conteúdo ofensivo ou não. A curva ROC (Característica de Operação do Receptor) é construída traçando a taxa de verdadeiros positivos em relação à taxa de falsos positivos para diferentes pontos de corte de classificação. A AUROC é a área sob essa curva, variando de 0 a 1, onde um valor mais próximo de 1 indica um modelo com melhor capacidade de distinguir entre as classes. Em geral, uma AUROC de 0,5 indica um modelo com desempenho aleatório, enquanto valores acima de 0,5 indicam um desempenho melhor do que o aleatório \cite{BradleyAUROC}.

\begin{figure*}[!htbp]
	\centering
	\includegraphics[scale=0.36]{imagens/auroc.png}
    \caption {Curva ROC para um classificador ``melhor'' e ``pior'' \cite{RocCurveWiki}.}
\end{figure*}

% bagging, boosting e stacking

O \textit{ensemble learning} é uma técnica amplamente adotada para melhorar a precisão e a estabilidade dos modelos de aprendizado de máquina, frequentemente utilizada em problemas do mundo real, onde a precisão é crucial.

\section{Processamento de Linguagem Natural}

O processamento de linguagem natural (PLN) é uma subárea da inteligência artificial e da linguística que estuda a compreensão e geração automática da linguagem humana. Os sistemas de geração de linguagem natural convertem informação armazenadas em bancos de dados computacionais em linguagem compreensível ao ser humano, enquanto os sistemas de compreensão de linguagem natural convertem ocorrências da linguagem humana em representações mais formais, mais facilmente manipuláveis por programas de computador. O PLN abrange uma variedade de tarefas, como a tradução e geração de respostas automáticas, sumarização de texto, reconhecimento de fala, análise de sentimento, que busca extrair qualidades subjetivas – emoções, humor, sarcasmo, ofensa, motivação, confusão – do texto e muito mais. O objetivo principal do PLN é permitir que os computadores compreendam, interpretem e gerem texto e fala de forma semelhante a linguagem humana\footnote{\textit{What is Natural Language Processing?} | IBM. Disponível em: \url{https://www.ibm.com/topics/natural-language-processing}. Acesso em: 12 de jun. de 2023.}.

% O PLN envolve uma série de técnicas e algoritmos que permitem que os computadores processem, analisem e compreendam textos escritos ou falados em uma linguagem natural. Essas técnicas incluem análise morfológica (separação de palavras em radicais, prefixos e sufixos), análise sintática (identificação da estrutura gramatical de uma sentença), análise semântica (extração de significado das palavras e frases) e análise pragmática (interpretação do contexto e intenção do usuário).

Atualmente, os modelos de aprendizagem profunda e as técnicas baseadas em redes neurais possibilitam sistemas de PLN serem capazes de aprender e extrair significados cada vez mais precisos de grandes quantidades de texto bruto. Esses sistemas têm a capacidade de aprimorar seu desempenho à medida que trabalham, permitindo uma compreensão mais avançada da linguagem natural, como os \textit{transformers}.

\subsection{\textit{Transformers}}

Os \textit{transformers} são uma arquitetura de rede neural profunda que revolucionaram o campo do processamento de linguagem natural. Ao contrário das redes neurais convencionais, os \textit{transformers} têm a capacidade de processar informações de uma sequência de entrada de forma simultânea, em vez de depender de um processamento sequencial. Isso permite uma maior paralelização, uma melhor compreensão do contexto global do texto e, consequentemente, uma redução do tempo de treinamento, o que resulta em uma melhoria significativa na eficiência em comparação com as redes neurais tradicionais \cite{GoogleTeamTransformers}. Isso é possível graças ao mecanismo de atenção, que permite que cada elemento da sequência interaja com todos os outros, fornecendo contexto relevante. Os \textit{transformers} têm sido amplamente adotados e são considerados o estado da arte\footnote{\textit{state-of-the-art}. Disponível em: \url{https://www.oxfordlearnersdictionaries.com/definition/english/state-of-the-art}. Acesso em: 12 de jun. de 2023.} no processamento de linguagem natural. Essa designação se refere à abordagem mais avançada e eficaz atualmente disponível no campo do processamento de linguagem natural, baseada em resultados de pesquisa, no desempenho superior em comparação com outras abordagens e na validação em estudos científicos e aplicações práticas.

A arquitetura dos \textit{transformers} é composta por um codificador e um decodificador, ambos contendo várias camadas. O \textbf{codificador} recebe uma sequência de entrada e gera uma representação contextualizada dessa sequência, é responsável por capturar as informações relevantes e as relações entre as palavras ou elementos da sequência. Por sua vez, o \textbf{decodificador} utiliza a representação contextual gerada pelo codificador juntamente com uma sequência de referência para gerar uma sequência de saída, é responsável por produzir a saída desejada com base no contexto e nas informações disponíveis. Ambos utilizam os mecanismos de atenção, autoatenção (do inglês, \textit{self-attention}), atenção de múltiplas cabeças (do inglês \textit{multi-head attention}) e a rede \textit{feed-foward}.

A ``atenção'' é um mecanismo fundamental no processamento de sequências, ela permite que um modelo se concentre em partes relevantes de uma sequência ao ponderar a importância de cada elemento em relação aos outros, calculando sua importância em relação a todos os outros elementos, resultando em uma matriz de pesos de atenção. Já a \textbf{autoatenção} em vez de comparar cada elemento com todos os outros, ela calcula as interações entre os elementos dentro da mesma sequência. Isso é feito por meio de transformações lineares e funções de similaridade, como o a função \textit{softmax}. A autoatenção permite que cada elemento da sequência se relacione com todos os outros elementos, capturando informações contextuais e de dependência importantes. 

Por sua vez, a atenção de múltiplas cabeças permite que o modelo capture informações contextuais de diferentes perspectivas, dividindo a representação de entrada em várias ``cabeças'', cada uma responsável por calcular pesos de atenção que indicam a importância relativa das palavras em relação umas às outras. Os valores são então combinados de acordo com os pesos de atenção, resultando em uma representação ponderada das palavras, que captura relações complexas e melhora a capacidade de representação e aprendizado dos \textit{transformers}.

Por fim, a rede \textit{feed-forward} é um tipo de rede neural, composta por duas camadas lineares com uma função de ativação não linear aplicada entre elas, que é executada individualmente a cada posição na sequência de entrada de forma idêntica. Seu objetivo é fornecer uma transformação não linear que captura padrões complexos e dependências entre os elementos da sequência. Ao introduzir a não linearidade e aprender padrões não lineares, a rede \textit{feed-forward} possibilita que o \textit{transformer} capte informações contextuais e complexas em cada posição da sequência. Isso contribui para a capacidade do modelo de processar efetivamente o contexto da linguagem natural e melhorar o desempenho.

%O codificador utiliza os mecanismos de autoatenção para calcular os pesos de importância de cada palavra em relação a todas as outras palavras da sequência. Essa abordagem permite que o codificador leve em consideração o contexto global ao processar a entrada. E o decodificador faz uso da atenção e da autoatenção quanto de atenção em relação à saída do codificador, para determinar a relevância das palavras durante o processo de geração da sequência de saída.

%No \textbf{codificador}, cada camada possui duas subcamadas: uma de autoatenção de várias cabeças (do inglê, \textit{multi-head self-attention}) e outra por uma rede \textit{feed-forward} totalmente conectada, e cada subcamada é seguida por uma conexão residual e normalização. No \textbf{decodificador}, também há duas subcamadas por camada: uma de autoatenção \textit{multi-head} e outra de atenção \textit{multi-head} que se concentra na saída do codificador. Além disso, o decodificador possui uma subcamada com uma rede \textit{feed-forward} totalmente conectada, também seguida por conexões residuais e normalização. A saída do decodificador passa por uma camada linear e uma função \textit{softmax} para gerar a sequência de saída final. Essa estrutura em camadas permite que o \textit{transformer} capture as dependências entre as palavras em uma sequência e produza resultados precisos em tarefas de processamento de linguagem natural, como tradução, sumarização e classificação de texto.

\begin{figure*}[!htbp]
	\centering
	\includegraphics[scale=0.6]{imagens/arch-transformers.png}
    \caption {Modelo de arquitetura de um \textit{transformer} \cite{GoogleTeamTransformers}.}
\end{figure*}


% No entanto, como um único vetor deve capturar toda a sequência de informações na arquitetura do codificador-decodificador, isso dificulta a retenção do conhecimento no início da série e a codificação de dependências de longo alcance. Em outras palavras, quando o codificador processa toda a sequência de entrada e comprime a informação em um vetor de contexto, ele frequentemente se esquece do início da entrada. Para resolver este problema, o mecanismo de atenção [5] foi estabelecido.

% Essa arquitetura é composta por múltiplas camadas de auto-atenção (self-attention) e camadas totalmente conectadas. O mecanismo de atenção permite que cada palavra em uma sequência interaja com todas as outras palavras, permitindo que o modelo capture informações contextuais mais ricas. Isso resulta em uma representação mais poderosa do texto e melhora o desempenho em tarefas de PLN.

% Transformer é composto de dois mecanismos separados: um codificador que lê o texto de entrada e um decodificador que gera a predição para a tarefa.

% Os Transformers consistem em duas partes principais: o codificador e o decodificador. O codificador recebe a sequência de entrada e a transforma em representações intermediárias, enquanto o decodificador gera a sequência de saída com base nessas representações.

Os \textit{transformers} têm sido amplamente adotados em várias aplicações de PLN e até mesmo em outros domínios, devido à sua capacidade de lidar com sequências de texto de comprimento variável, aprender representações contextualizadas e gerar previsões precisas. Comparados aos modelos de redes neurais, os transformadores são mais receptivos à paralelização, permitindo o treinamento em conjuntos de dados maiores. Isso levou ao desenvolvimento de sistemas pré-treinados, como o BERT e o GPT, que foram treinados com grandes conjuntos de dados de linguagem. Além disso, os \textit{transformers} têm mostrado sucesso não apenas no processamento de linguagem natural, mas também em outras áreas, como visão computacional.

\subsection{BERT}

% modelo pré-treinado fornecido pela biblioteca HuggingFace Transformers

O BERT (do inglês, \textit{Bidirectional Encoder Representations from Transformers}) é um modelo de linguagem pré-treinado desenvolvido pela \citeonline{BERTImagem}, baseado na arquitetura do \textit{transformer}. Foi treinado em uma grande quantidade de dados não rotulados para aprender representações ricas e contextuais das palavras. O treinamento do BERT envolve duas tarefas principais: ``previsão de máscara'' e ``próxima sentença''. Na previsão de máscara, palavras em uma sequência de texto são mascaradas e o modelo é treinado para prever as palavras mascaradas com base no contexto. Na tarefa de próxima sentença, o BERT aprende a prever se duas sentenças são sequenciais ou não.

O BERT alcançou resultados significativos em várias tarefas de processamento de linguagem natural quando combinado com tarefas de ajuste fino em conjuntos de dados específicos. Ao pré-treinar o modelo em grandes quantidades de dados e depois ajustá-lo para tarefas específicas, o BERT é capaz de capturar nuances semânticas e sintáticas das palavras, melhorando o desempenho em classificação de texto, resposta a perguntas, sumarização e muito mais.

\begin{figure*}[!htbp]
	\centering
	\includegraphics[scale=0.4]{imagens/BERT.png}
    \caption {Representação do BERT (bidirecional), e da OpenAI GPT (unidirecional) \cite{BERTImagem}.}
\end{figure*}

O diferencial do BERT em relação a outros modelos de linguagem pré-treinados é a sua capacidade bidirecional de compreender o contexto das palavras, considerando o contexto anterior e posterior de cada palavra ao realizar o seu processamento, o que permite ao modelo uma melhor compreensão do significado das palavras e frases em um texto, levando em conta as dependências entre elas. Os resultados mostram que modelos de linguagem treinados de forma bidirecional, como o BERT, fornecem uma compreensão mais profunda do contexto linguístico do que os modelos unidirecionais.

%

\section{Visão Computacional}

A visão computacional é uma área de estudo e aplicação da inteligência artificial que busca capacitar os computadores a interpretarem, analisarem e compreenderem o conteúdo visual de imagens ou vídeos de maneira similar ao ser humano. Isso envolve o desenvolvimento de algoritmos e técnicas que permitem que os sistemas computacionais obtenham informações úteis a partir de dados visuais, e realizem tarefas como reconhecimento facial e de objetos, detecção de padrões, análise de expressões faciais, classificação automática de imagens em redes sociais entre outros\footnote{\textit{What is Computer Vision?} | IBM. Disponível em: \url{https://www.ibm.com/topics/computer-vision}. Acesso em: 12 jun. 2023.}.

A integração de técnicas de visão computacional com outros campos, como processamento de linguagem natural, tem impulsionado pesquisas e avanços significativos na análise e compreensão de dados visuais. Essa integração tem permitido a aplicação de técnicas avançadas de processamento de linguagem natural no campo da visão computacional, resultando em melhorias nas tarefas relacionadas à interpretação e compreensão de informações visuais, como o \textit{VisualBERT}.

% A informação visual é processada usando técnicas de visão computacional, como detecção de objetos e extração de recursos. O modelo é então treinado para prever se um determinado meme contém discurso de ódio ou não, com base nas informações textuais e visuais.

% dada a grande melhoria nas tarefas de PNL, seu uso se estendeu à visão computacional [105] e pesquisas adicionais sobre atenção lançaram uma luz sobre diferentes mecanismos de atenção

\subsection{\textit{VisualBERT}}

% pré-treinamento

O \textit{VisualBERT} é um modelo de linguagem pré-treinado que combina processamento de linguagem natural e a visão computacional para entender e interpretar informações visuais e textuais simultaneamente, baseado na arquitetura do BERT, ao contrário dos modelos que se baseiam apenas em texto, ele incorpora também representações visuais das imagens para melhorar o desempenho em tarefas que envolvem texto e contexto visual. O modelo é treinado em um grande conjunto de dados multimodais que combina texto descritivo e imagens correspondentes, tal abordagem permite que o \textit{VisualBERT} capture informações visuais relevantes, permitindo uma melhor compreensão de textos relacionados a imagens e aprimorando o desempenho em tarefas como descrição de imagens, geração de legendas entre outros. O \textit{VisualBERT} representa uma extensão dos modelos de linguagem pré-treinados, integrando efetivamente a visão computacional ao processamento de linguagem natural para uma análise mais completa e precisa de conteúdos multimodais \cite{VisualBERTArt}.

O \textit{VisualBERT} é pré-treinado usando dois objetivos semelhantes aos usados no BERT: modelagem de linguagem mascarada com previsão de alinhamento de imagens e legendas. O processo de pré-treinamento envolve treinar o modelo com um grande conjunto de dados multimodais que combinam texto descritivo e imagens correspondentes, para prever alguma parte oculta (mascarada) da entrada, seja uma parte de uma imagem ou uma palavra do texto. Após o pré-treinamento, o modelo passa por um processo de ajuste fino em conjuntos de dados específicos para aprimorar seu desempenho. Isso permite que o modelo se adapte às características e nuances das tarefas específicas, refinando ainda mais suas representações multimodais.

A arquitetura do \textit{VisualBERT} utiliza uma camada de entrada compartilhada para codificar tanto o texto quanto as representações visuais. Utilizando técnicas de codificação de palavras e redes neurais, essa camada captura informações contextuais das palavras e das representações visuais por meio de várias camadas do codificador do \textit{transformer}, assim como o BERT. Cada camada do codificador possui subcamadas de atenção \textit{multi-head} e redes \textit{feed-forward}, permitindo a captura de interações e padrões entre palavras e informações visuais. Os recursos visuais extraídos das propostas de objetos são tratados como \textit{tokens} de entrada não ordenados e incorporados no \textit{VisualBERT} juntamente com o texto. Essa abordagem permite que o modelo capture e relacione informações tanto textuais quanto visuais, permitindo uma compreensão mais completa dos dados multimodais.

% Durante o pré-treinamento, o VisualBERT é treinado em tarefas de máscara de palavras e previsão de sentença seguintes. Essa abordagem permite que o modelo aprenda representações ricas e contextuais para texto e imagens. Além disso, o modelo é ajustado para diferentes tarefas específicas, capturando informações contextuais entre texto e imagem e permitindo uma compreensão completa dos dados multimodais.

\begin{figure*}[!htbp]
	\centering
	\includegraphics[scale=0.35]{imagens/visualBERT.png}
    \caption {Arquitetura do \textit{VisualBERT} \cite{VisualBERTArt}.}
\end{figure*}

% Ele utiliza uma representação compartilhada que incorpora características visuais e contextos linguísticos, permitindo que o modelo compreenda a relação entre palavras e elementos visuais presentes em uma imagem.

% discute o processo de treinamento para modelos de aprendizado profundo multimodal. A seção então descreve o esquema de treinamento para modelos multimodais, que envolve pré-treinamento em tarefas multimodais intermediárias ou proxy antes do ajuste fino na tarefa multimodal em questão. A seção menciona vários modelos pré-treinados que foram propostos, incluindo ViLBERT, VisualBERT, LXMERT, VL-BERT, OSCAR e UNITER. Esses modelos têm esquemas de treinamento muito semelhantes, que envolvem pré-treinamento em grandes conjuntos de dados de legendas de imagens. A seção conclui destacando a importância do pré-treinamento e do ajuste fino no aprendizado profundo multimodal, que permite que os modelos aprendam representações ricas de dados visuais e textuais.

\section{Multimodalidade}

A multimodalidade refere-se à capacidade de combinar e integrar múltiplas modalidades de comunicação, como texto, imagem, som, gestos e outros elementos sensoriais, para transmitir ou compreender informações de maneira mais rica e completa\footnote{\textit{What is Multimodal?} | \textit{University of Illinois Springfield}. Disponível em: \url{https://www.uis.edu/learning-hub/writing-resources/handouts/learning-hub/what-is-multimodal}. Acesso em: 12 jun. 2023.}. É a interação entre diferentes formas de representação e expressão que enriquece a comunicação e permite uma compreensão mais abrangente do conteúdo, permitindo uma comunicação mais rica e eficaz entre humanos e máquinas.

Ao lidar com a multimodalidade, os sistemas computacionais buscam interpretar e extrair informações de diferentes modalidades, permitindo que os usuários se envolvam de maneira mais natural e efetiva com a tecnologia. Essa abordagem é amplamente aplicada em áreas como processamento de linguagem natural, visão computacional, interação humano-computador e aprendizado de máquina, onde a combinação de modalidades permite uma compreensão mais rica e abrangente dos dados. Ao combinar informações de várias modalidades, é possível melhorar o desempenho de uma variedade de tarefas, resultando em soluções mais eficazes para uma ampla gama de desafios.

% Houve uma onda de interesse em problemas multimodais desde 2015 em resposta a perguntas visuais [5, 6], legendas de imagens [7, 8], reconhecimento de fala [9, 10] e além. https://arxiv.org/pdf/2012.12975.pdf

\section{Trabalhos Relacionados}

% apresentar antes de iniciar os tópicos

\subsection{Desafio dos Memes Odiosos (\textit{The Hateful Memes Challenge})} 

Este trabalho realizado pela equipe da Meta AI \cite{ArticleHatefulMemesChallenge2021} propôs um desafio para a classificação multimodal, com foco na detecção de discurso de ódio em memes multimodais. Fornecem modelos multimodais com diferentes níveis de sofisticação e ilustram como foi feita a criação do banco de dados com os memes odiosos e os memes categorizados como “confundidores benignos”. Esses confundidores foram adicionados ao conjunto de dados para aumentar a dificuldade do desafio, pois usam as mesmas imagens e frases que os memes odiosos, mas em contextos que não se enquadram como discurso de ódio. O objetivo deste desafio foi estimular a inovação e impulsionar o progresso no raciocínio e compreensão multimodal, que podem ter impactos positivos em várias tarefas e aplicações, ao mesmo tempo em que facilita o progresso da detecção de discurso de ódio e seu combate por meio de métodos automáticos.

\subsection{Detectando Discurso de Ódio em Memes Usando Abordagens de Aprendizado Profundo Multimodal} 

Os participantes do desafio proposto pela Meta IA adotaram uma abordagem que consistiu, primeiramente, em expandir o conjunto de treinamento por meio da busca por conjuntos de dados semelhantes na web. Em seguida, foram extraídas características das imagens utilizando algoritmos de detecção de objetos (\textit{Detectron}), o que permitiu aprimorar o modelo pré-treinado (\textit{VisualBERT}). Por fim, foi realizada a pesquisa de hiperparâmetros e aplicada a técnica de voto majoritário, atingindo 0,811 AUROC com uma precisão de 0,765 no conjunto de teste do desafio \cite{velioglu2020hateful}.
%\chapter{Metodologia}
\label{cap:03}
%\chapter{Desenvolvimento}
\label{cap:04}

%\input{cap5resultados}
%\input{cap6conclusao}

% ----------------------------------------------------------
% ELEMENTOS PÓS-TEXTUAIS
% ----------------------------------------------------------
\postextual
% ----------------------------------------------------------

% ----------------------------------------------------------
% Referências bibliográficas
% ----------------------------------------------------------
\bibliography{referencias}

% ----------------------------------------------------------
% Glossário
% ----------------------------------------------------------
%
% Consulte o manual da classe abntex2 para orientações sobre o glossário.
%
%\glossary

% ----------------------------------------------------------
% Apêndices
% ----------------------------------------------------------

% ---
% Inicia os apêndices
% ---
%\input{x07apendices}
% ---


% ----------------------------------------------------------
% Anexos
% ----------------------------------------------------------

% ---
% Inicia os anexos
% ---
%\input{x08anexos}

%---------------------------------------------------------------------
% INDICE REMISSIVO
%---------------------------------------------------------------------
%\phantompart
\printindex
%---------------------------------------------------------------------

\end{document}
